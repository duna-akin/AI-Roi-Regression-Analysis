\documentclass[11pt]{article}

\usepackage[utf8]{inputenc}
\usepackage{amsmath}
\usepackage{geometry}
\usepackage{hyperref}
\usepackage{setspace}
\geometry{margin=1in}

\begin{document}
\doublespacing
\vspace*{-2cm} 
\begin{center}
    {\LARGE \textbf{AI Investment and Business Returns}}\\[0.5em]
    {\large Correlation or Causation?}\\[1em]
    Tuna Akın\\
    \href{https://mlq.ai/media/quarterly_decks/v0.1_State_of_AI_in_Business_2025_Report.pdf}{State of AI in Business 2025 by MIT NANDA}\\
    \today
\end{center}

\section*{Summary}
The report documents \$30--40B in enterprise GenAI spending yet finds that about 95\% of organizations see no measurable ROI; only around 5\% of task-specific tools reach production and impact P\&L. The bottleneck is not model quality or regulation but poor workflow integration and lack of learning/memory in systems. Spending skews toward sales and marketing because outcomes are visible, while higher-ROI gains often arise in back-office automation (finance, operations) that are underfunded. Sector patterns matter: meaningful disruption concentrates in technology and media, while seven of nine sectors show little structural change.

\section*{Regression Model}
We can summarize the article’s headline relationship with a simple bivariate model:
\[
ROI_i = \beta_0 + \beta_1 \, InvestmentAI_i + \varepsilon_i,
\]
where $ROI_i$ is firm $i$’s return on investment and $InvestmentAI_i$ is spending on AI initiatives.

\section*{Causality Concerns}
The article may suggest (or readers may infer) that more AI spending \emph{causes} higher ROI. However, $InvestmentAI_i$ is unlikely exogenous:

\begin{itemize}
    \item \textbf{Industry mix:} Tech/media firms both invest more and show higher ROI; pooling sectors biases $\hat{\beta}_1$ upward if high-ROI sectors are heavier investors.
    \item \textbf{Management quality / readiness:} Adaptive firms invest early and execute better, raising ROI even absent AI (omitted variable bias).
    \item \textbf{Implementation strategy:} External partnerships reach deployment about twice as often as internal builds; strategy is a mediator between investment and ROI.
    \item \textbf{Spend composition:} Budgets favor visible front-office tools, while ROI often comes from back-office automation; misallocation can attenuate or reverse returns at a given spend level.
\end{itemize}

Directionally, selection on strong firms (and favorable sectors) tends to \emph{inflate} $\hat{\beta}_1$ (positive bias). Yet the report’s central fact—that most firms realize zero ROI despite substantial spend—implies the naive correlation exaggerates the average causal effect of investment on ROI for typical adopters.

\section*{Conclusion}
“More AI spend $\rightarrow$ higher ROI” is, at best, a fragile correlation. Causality hinges on \emph{how} capital is deployed: workflow integration, learning/memory, sector context, and build-vs-buy strategy. Only targeted, process-specific deployments deliver reliable ROI; broad spending alone does not.

\end{document}
